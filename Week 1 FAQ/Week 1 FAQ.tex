\documentclass{article}

\usepackage[utf8]{inputenc}
\usepackage[english]{babel}
\usepackage{xcolor,upquote,charter}
\usepackage[hidelinks]{hyperref}
\usepackage[top=3cm, bottom=4cm, left=3cm, right=3.5cm]{geometry}
\usepackage{caption}

\usepackage{minted}
% \usemintedstyle{monokai}

\newcommand{\inlinecode}[1]{\mintinline{python}{#1}}
\newcommand{\link}[2]{\textcolor{blue}{\href{#2}{#1}}}
\newcommand{\question}[1]{\item[$\bullet$ #1] \hfil}

\newenvironment{answer}{}{}
\newenvironment{faq}{\begin{description}}{\end{description}}

\definecolor{bg}{rgb}{0.95,0.95,0.95}

\setminted[python]
{
	mathescape,
	python3=true,
	tabsize=4,
	bgcolor=bg
}

\pagestyle{empty}

\captionsetup[table]{name=Listing}

\begin{document}
	
	\section*{\Huge6.00.1x Week 1 FAQ}
	
		\subsection*{\LARGE Commonly Used Acronyms}
		
			\begin{description}
				\item[PSETs] Problem Sets
				
				\item[Fexes] Finger Exercises
				
				\item[Community TAs/TAs] Community Teaching Assistants
			\end{description}
		
		\subsection*{\LARGE Course Logistics}
		
			\begin{faq}
				\question{Can I try the course before upgrading to a Verified Certificate\,?}
				
				\begin{answer}
					Of course! You are welcome to try the course and all its materials before
					upgrading to a verified certificate but if you wish to receive a certificate you
					must upgrade before the date indicated in the ``Home'' tab.
				\end{answer}
				
				\question{Are finger exercises graded\,?}
				
				\begin{answer}
					Yes, completing finger exercises counts towards your final grade but you
					have unlimited attempts to complete them. Finger exercises are here to help
					you solidify your knowledge and learn from your mistakes.
				\end{answer}
				
				\question{How many attempts am I allowed for finger exercises\,?}
				
				\begin{answer}
					As many as you need. Finger exercises are here to help you solidify your
					knowledge, detect areas where you need to work on and learn the topics
					presented.
				\end{answer}
				
				\question{Can I spend more time than the estimated time on an exercise\,?}
				
				\begin{answer}
					Absolutely! It’s okay to spend more than the estimated time on a finger
					exercise, just make sure that you understand the topic and what the exercise
					required you to do so you are able to solve it more quickly the next time you
					are presented with a similar challenge. If you have any questions on the
					exercises please ask them in the discussion forums. Your classmates and
					Community TAs will be there to help you.
				\end{answer}
				
				\question{Are there deadlines I need to meet to complete lectures and exercises\,?}
				
				\begin{answer}
					No, in this self-paced version of the course there are no deadlines for lectures
					and exercises but you must complete all assignments before the course
					ends. You can check this date on the ``Home'' tab.
				\end{answer}
				
				\question{How to submit code to the grader where it is said the following sentences or alike\,?}
				
				\textcolor[HTML]{900090}{Assume that variables are already declared} or \textcolor[HTML]{900090}{Do not ask for user input}.
				
				\begin{answer}
					Don't include the variable declaration when you paste your code to the
					grader, remove that line from your code as in the example \textcolor{blue}{\ref{Code you run on your IDE to test your solution}} and \textcolor{blue}{\ref{Code submitted to the grader}}:
					
					\begin{table}[htb]
						\caption{Code you run on your IDE to test your solution}
						\label{Code you run on your IDE to test your solution}
						\begin{minted}[linenos=true,stripnl=false]{python}
						# your own test value
						num = 5
						# what you are to code
						if num != 6:
						    print(num)  
						\end{minted}
					\end{table}
					
					\begin{table}[htb]
						\caption{Code submitted to the grader}
						\label{Code submitted to the grader}
						\begin{minted}[linenos=true, firstnumber=3]{python}
						# what you are to code
						if num != 6:
						    print(num)
						
						\end{minted}
					\end{table}
				\end{answer}
				
				\question{How can I keep indentation consistent when I paste my code to the grader\,?}
				
				\begin{answer}
					It usually helps to use tabs instead of spaces to create indentation.
					If you are using Sublime Text as you Text Editor, highlight your code to check if you see
					these dots to the left of your indented code:
					
					
					
					If you do, this means that you are using spaces instead of tabs for
					indentation. To change this, select your code and click on the option highlighted with
					the red arrow on the image below, select the option located at the bottom right corner
					of the screen that says “spaces”.
					
					
					
					Then click on “Convert Indentation to Tabs”
					
					
					
					This will convert the spaces to tabs for the code you selected. After this you
					can copy and paste your code to the grader to keep your indentation. Good luck and
					if you have any questions we are here to help you in the discussions forums!
				\end{answer}
			\end{faq}
		
		\subsection*{\LARGE Course Content}
		
			\begin{faq}
				
				\question{What is Static Semantic vs. Semantic\,?}
				
				\begin{answer}
					Static semantics checks which syntactically correct expressions have
					meaning (e.g $3 \times 5$ is syntactically correct and multiplying two numbers has
					meaning) while Semantics interprets the meaning of the expression. That
					meaning can't be ambiguous.
				\end{answer}
				
				\question{Why is True different from true and False different from false in Python\,?}
				
				\begin{answer}
					Python is case sensitive; therefore the correct way to write Booleans is \textbf{True}
					and \textbf{False}. Using true and false will return an error because they are not
					recognized as Booleans.
				\end{answer}
				
				\question{What is Bool\,?}
				
				\begin{answer}
					Bool is a way to refer to Booleans, the truth values True and False.
				\end{answer}
				
				\question{Difference between $/$ and $//$ in Python}
				
				\begin{answer}
					The $/$ operator returns a float, while $//$ returns an int.
				\end{answer}
				
				\question{What does \inlinecode{!=} mean\,?}
				
				\begin{answer}
					This is a comparison operator that means ``not equal''. It returns True if the
					values compared are NOT equal and returns False if the values compared
					are equal.
				\end{answer}
				
				\question{What is the syntax for advanced string slicing\,?}
				
				\begin{answer}
					\inlinecode{"string"[start: end: step]}
				\end{answer}
				
				\question{Does \inlinecode{input()} always return a string\,?}
				
				\begin{answer}
					Yes. The Built-in function \inlinecode{input()} casts (``converts'') the input into a string and
					returns it. You can find more information on this function at \link{this link}{https://docs.python.org/3/library/functions.html\#input}.
				\end{answer}
				
				\question{What does the error ``\,'str' object is not callable'' mean\,?}
				
				\begin{answer}
					This error occurs when you name one of your variables \inlinecode{str}. If you use this
					name, you will ``replace'' the built-in function \inlinecode{str()} and you will not be able to
					use it anymore as a built-in function (More information on \inlinecode{str} at \link{this link}{https://docs.python.org/3/library/functions.html\#func-str}).
				\end{answer}
				
				\question{Difference between \inlinecode{if} and \inlinecode{elif}}
				
				\begin{answer}
					\inlinecode{if} is the first condition checked in an if/elif/else statement. If it is True the ``if''
					code block will be executed. Else, if it is False the subsequent ``elif'' conditions
					will be checked. The code block of the first \inlinecode{elif} condition that evaluates to
					True will be executed. If none of them evaluates to True, the \inlinecode{else} code block
					will be automatically executed.
				\end{answer}
				
				\question{What is \# and what is it used for\,?}
				
				\begin{answer}
					In Python, \# is used to comment out a line of code or to add a comment to
					your code. These lines will not be executed as code, they are simply a way to
					help you explain your thought process to yourself and to other developers.
					For comments you do not need to follow the syntax of the programming
					language, you can write them as if you were talking to yourself or explaining what
					your code does. It is a very helpful way to communicate with developers that will look
					at your code in the future to try to understand how it works.
				\end{answer}
				
				\question{What is the difference between an Error and an Infinite Loop\,?}
				
				\begin{answer}
					An error is a warning thrown by your IDE or Shell, it is descriptive and gives
					you some information on what caused it. An infinite loop occurs when a while
					loop keeps executing indefinitely because there is no condition to stop its
					execution or if there is, that condition is never met. Infinite loops may not
					throw errors at all and the code will keep executing indefinitely until you
					manually execute a command to stop its execution.
				\end{answer}
				
				\question{How do I stop an infinite loop\,?}
				
				\begin{answer}
					By pressing \texttt{CTRL + C} on Windows.
				\end{answer}
				
				\question{Difference between While and For loop}
				
				\begin{answer}
					A While loop is used for situations where we want to keep executing a code
					block while a condition is not met, meaning that we don't know ahead of time
					how many times the loop has to be executed to meet that condition. In
					contrast, a For Loop is used when we do know how many times the loop has
					to be executed.
				\end{answer}
				
				\question{What is \inlinecode{str}\,?}
				
				\begin{answer}
					\inlinecode{str} is the built-in string class. You will learn about classes in the next few
					weeks of the course but right now you need to know that you can check if a
					variable is a string by comparing it directly to \inlinecode{str} (To learn more about \link{str}{https://docs.python.org/3/library/functions.html\#func-str})
				\end{answer}
				
				\question{Difference between \inlinecode{for i in} and \inlinecode{for i in range}\,?}
				
				\begin{answer}
					\inlinecode{for i in} is used to iterate over all the elements of a sequence (characters in a
					string, and elements of other data structures that you will learn in the next
					weeks of the course such as tuples and list). \inlinecode{for i in range} is a way to
					execute the loop a specific number of times that you can customize. Both
					update their variable (\inlinecode{i} in this case) on every loop iteration.
				\end{answer}
				
				\question{How does \inlinecode{while True} work?}
				
				\begin{answer}
					\inlinecode{while True} will execute a While loop indefinitely until it reaches a break
					statement to stop its execution. This is useful when you need to perform the
					same task an indefinite number of times until a condition is met.
				\end{answer}
				
			\end{faq}
	
\end{document}
